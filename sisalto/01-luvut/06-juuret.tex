\chapter{Juuret}

\section{Neliöjuuri}

Ajatellaan, että neliön pinta-ala on $a$. Halutaan tietää, mikä on kyseisen neliön sivun pituus. Vastausta tähän kysymykseen kutsutaan luvun $a\ge 0$ neliöjuureksi ja merkitään $\sqrt{a}$. Luvun $a$ neliöjuuri on myös yhtälön $x^2 = a$ ratkaisu. Tällöin täytyy kuitenkin huomata, että myös luku $x=-\sqrt{a}$ toteuttaa kyseisen yhtälön. Neliöjuurella tarkoitetaan kyseisen yhtälön epänegatiivista ratkaisua. Tämä on luonnollista, koska neliön sivun pituus ei voi olla negatiivinen luku.

\laatikko{Luvun $a$ neliöjuuri on epänegatiivinen luku, jonka neliö on $a$. Tämä voidaan ilmaista myös $(\sqrt{a})^2=a$.}

Neliöjuurta ei tällä kurssilla määritellä negatiivisille luvuille, koska neliön pinta-ala on aina positiivinen luku tai nolla. Käytännössä lukujen neliöjuuria lasketaan usein laskimella.


\begin{esimerkki}
Laske.
\begin{enumerate}[a)]
\item $\sqrt{4}$

\item $\sqrt{144}$

\item $\sqrt{3471}$.
\end{enumerate}

{\bf Ratkaisut.}

a)
Laskimella tai päässä laskemalla saadaan, että $\sqrt{4} = 2$, koska $2\geq0$ ja $2^2 =4$.

b) 
$\sqrt{144}=12$. Tämä voidaan vielä tarkistaa laskemalla $12^2 = 12\cdot 12=144$.

c)
$\sqrt{4471}\approx 66,9$. Vertailun vuoksi laskimella saadaan myös $67\cdot 67=4489$.

{\bf Vastaukset.}
a) $2$, b) $12$, c) $66,9$.

\end{esimerkki}

\begin{esimerkki}
Taulutelevision kooksi (halkaisijaksi) on ilmoitettu mainoksessa $46,0$ tuumaa ($116,8$ cm) ja kuvasuhteeksi 16:9. Kuinka leveä televisio on (senttimetreinä)?

{\bf Ratkaisu.}

Taulutelevision halkaisija, alareuna ja toinen sivu muodostavat suorakulmaisen kolmion. Kolmion hypotenuusa on television halkaisija ja kateetit alareuna ja toinen sivu.

Kuvasuhteen perusteella kateettien pituuksia voidaan merkitä $16x$ ja $9x$. Pythagoraan lauseesta saadaan
\[
(116,8)^2 = (16x)^2 + (9x)^2
\]
eli
\[
13642,24 = (256+81)x^2.
\]
Siten
\[
x^2 = \frac{13642,24}{337}
\]
ja siis
\[
x= \sqrt{\frac{13642,24}{337}} \approx 6,36.
\]
Television leveys on noin $16x = 16\cdot 6,36\approx 102$ cm.

{\bf Vastaus.} Noin $102$ cm.
\end{esimerkki}



\section{Kuutiojuuri}

Kuution tilavuus on $a$. Halutaan tietää, mikä on kyseisen kuution sivun pituus. Vastausta tähän kysymykseen kutsutaan luvun $a\ge 0$ kuutiojuureksi ja merkitään $\sqrt[3]{a}$. Luvun $a$ kuutiojuuri on myös yhtälön $x^3 = a$ vastaus.

\laatikko{Luvun $a$ kuutiojuuri on luku, jonka kuutio $a^3$ on $a$. Tämä voidaan ilmaista myös $(\sqrt[3]{a})^3=a$.}

Jos $a<0$, niin kysymys kuutiosta jonka tilavuus on $a$, ei ole mielekäs. Tästä huolimatta kuutiojuuri määritellään myös negatiivisille luvuille. Syy tähän on, että yhtälöllä $x^3=a$ on tässäkin tapauksessa yksikäsitteinen ratkaisu. Positiivisen luvun kuutiojuuri on aina positiivinen ja negatiivisen luvun kuutiojuuri on aina negatiivinen luku. Kuutiojuuren voi siis ottaa mistä tahansa reaaliluvusta.


\begin{esimerkki}
Laske.
\begin{enumerate}[a)]
\item $\sqrt[3]{27}$

\item $\sqrt[3]{1397}$

\item $\sqrt[3]{2197}$.
\end{enumerate}

{\bf Ratkaisut.}

a)
Laskimella tai päässä laskemalla saadaan, että $\sqrt[3]{27} = 3$, koska  $3^3 =3\cdot 3\cdot 3=27$.

b) 
$\sqrt[3]{1397}\approx 11,18$. 

c)
$\sqrt[3]{2197}=13$.
Tämä voidaan vielä tarkistaa laskemalla $13^3 = 13\cdot 13\cdot 13=2197$.

{\bf Vastaukset.}

a) $3$, b) $11,18$, c) $13$.
\end{esimerkki}


\section{Korkeampia juuria}


Yhtälön $x^n=a$ ratkaisujen avulla voidaan määritellä $n$:s juuri mille tahansa positiiviselle kokonaisluvulle $n$. Neliö- ja kuutiojuurten tapauksesta voidaan kuitenkin voi huomata, että kuutiojuuri on määritelty kaikille luvuille, mutta neliöjuuri vain epänegatiivisille luvuille. Tämä toistuu myös muissa juurissa: parilliset ja parittomat juuret on määriteltävä erikseen.

Juurimerkinnällä $\sqrt[n]{a}$ (luetaan \emph{$n$:s juuri} luvusta $a$) tarkoitetaan lukua, joka toteuttaa ehdon $(\sqrt[n]{a})^n = a$. Jotta juuri olisi yksikäsitteisesti määritelty asetetaan lisäksi, että parillisessa tapauksessa $n$:s juuri tarkoittaa kyseisen yhtälön epänegatiivista ratkaisua.
%($\sqrt{a}, \sqrt[4]{a}, \sqrt[6]{a}$\ldots) vaadittava, että $b\ge0$.



\laatikko{
{\bf $n$:s juuri}

Luvun $a$ $n$:s juuri on epänegatiivinen luku, jonka $n$:s potenssi on $a$. Tämä voidaan ilmaista myös $(\sqrt[n]{a})^n=a$.}

Luvun toista juurta, eli neliöjuurta $\sqrt[2]{a}$ merkitään myös $\sqrt{a}$.

\laatikko{
{\bf Parillinen juuri.}

Luvun luvun $a$ parillinen juuri on $n$:s juuri, kun $n$ on parillinen luku:
$(\sqrt[n]{a})^n=a$, $a\ge0$.
}

Kuten kuutiojuuren tapauksessa, parittomat juuret määritellään kaikille reaaliluvuille $a$.

\laatikko{
{\bf Pariton juuri.}

Luvun luvun $a$ pariton juuri on $n$:s juuri, kun $n$ on pariton luku:
 $(\sqrt[n]{a})^n$, kaikilla $a\in \mathbb{R}$.
}

Korkeampien juurten laskeminen tapahtuu tavallisesti laskimella.

\begin{esimerkki}
Laske.
\begin{enumerate}[a)]
\item $\sqrt[4]{256}$
\item $\sqrt[5]{-243}$
\item $\sqrt[4]{-8}$
\end{enumerate}

{\bf Ratkaisut.}

a) Laskimella saadaan $\sqrt[4]{256}=4$ koska $4^2=256$.

b) Laskimella $\sqrt[4]{-243}=-3$, koska $(-3)^5=-243$.

c) Luvun $-8$ neljäs juuri $\sqrt[4]{-8}$ ei ole määritelty, koska minkään luvun neljäs potenssi ei ole negatiivinen.
\end{esimerkki}

\section*{Tehtäviä}

%Antti, lisää tämä
\begin{tehtava}
Laske seuraavat neliöjuuret laskimella:\\ 
a) $\sqrt{320}$,\ b) $\sqrt{15}$,\ c) $\sqrt{71}$
\begin{vastaus}
a) $17{,}89$ b) $3{,}87$ c) $8{,}43$
\end{vastaus}
\end{tehtava}


\begin{tehtava}
Laske $\sqrt{8100}$ päässä ajattelemalla juurrettava luku sopivana tulona.
\begin{vastaus}
$\sqrt{8100}=\sqrt{81}\sqrt{100}=9\cdot 10$.
\end{vastaus}
\end{tehtava}


\begin{tehtava}
Etsi luku $a>0$ jolle $a^4=83521$.
\begin{vastaus}
$a=\sqrt{\sqrt{83521}}$.
\end{vastaus}
\end{tehtava}


\begin{tehtava}
Oletetaan että suorakaiteen leveyden suhde korkeuteen on $2$ ja suorakaiteen pinta-ala on $10$. Mikä on suorakaiteen 
leveys ja korkeus?
\begin{vastaus}
Suorakaide muodostuu kahdesta vierekkäisestä neliöstä, joiden pinta-ala on $5$. Tämän neliön sivun pituus on $\sqrt{5}$.
Siis suorakaiteen korkeus on $\sqrt{5}$ ja leveys $2\sqrt{5}$.
\end{vastaus}
\end{tehtava}


\begin{tehtava}
Ajatellaan suorakulmaista hiekkakenttää, jonka pinta-ala on aari ($100^2 \, \mathrm{m}^2$). Lyhyemmän ja pidemmän sivujen 
pituuksien suhde on $4:3$. Laske Pythagoraan lauseen avulla matka hiekkakentän kulmasta kauimmaisena olevaan kulmaan.
\begin{vastaus}
$\frac{4}{3}x^2=100$ joten $x = \sqrt{\frac{300}{4}}$. 
Hypotenuusa: $\sqrt{x^2 + (\frac{4}{3}x)^2}=\sqrt{\frac{300}{4}+\frac{16}{9}\cdot \frac{300}{4}}
=\frac{5}{3}\sqrt{\frac{300}{4}}=14{,}43$.
\end{vastaus}
\end{tehtava}


\begin{tehtava}
Laske.
a) $\sqrt{64}$ \quad b) $\sqrt{-64}$ \quad c) $\sqrt[3]{64}$ \quad d) $\sqrt[3]{-64}$

\begin{vastaus}
a) 8 b) Ei määritelty c) 4 d) -4
\end{vastaus}
\end{tehtava}

\begin{tehtava}
Laske.
a) $\sqrt[4]{81}$ \quad b) $\sqrt[4]{-81}$ \quad c) $\sqrt[5]{32}$ \quad d) $\sqrt[5]{-32}$

\begin{vastaus}
a) 3 b) Ei määritelty c) 2 d) -2
\end{vastaus}
\end{tehtava}

\begin{tehtava}
Laske luvun $10$ potensseja: $10^1, 10^2, 10^3, 10^4, \ldots$ Kuinka monta nollaa on luvussa $10^n$? Laske sitten $\sqrt[6]{1~000~000}$ ja $\sqrt[10]{10~000~000~000}$.

\begin{vastaus}
$10^1 = 10, 10^2 = 100, 10^3 = 1~000, 10^4 = 10~000$. Luvussa $10^n$ on $n$ kappaletta nollia. Niinpä $\sqrt[6]{1~000~000} = 10$ ja $\sqrt[10]{10~000~000~000} = 10$.
\end{vastaus}
\end{tehtava}

\begin{tehtava}
Onko annettu juuri määritelty kaikilla luvuilla $a$? Millaisia arvoja juuri voi saada luvusta $a$ riippuen?\\
a) $\sqrt[4]{a^2}$ \quad b) $\sqrt[4]{-a^2}$ \quad c) $\sqrt[4]{(-a)^2}$ \quad d) $- \sqrt[4]{a^2}$

\begin{vastaus}

\begin{enumerate}[a)]
	\item Juuri on määritelty kaikilla luvuilla $a$, koska kaikkien lukujen neliöt ovat vähintään nolla. Vastaus on aina epänegatiivinen.
	\item Juuri on määritelty vain luvulla $a = 0$. Muilla $a$:n arvoilla $-a^2$ on negatiivinen, jolloin parillinen juuri ei ole määritelty. Ainoa vastaus, joka voidaan saada, on siis $\sqrt[4]{0} = 0$.
	\item Juuri on määritelty kaikilla luvuilla $a$, koska $(-a)^2$ on aina vähintään nolla. Vastaus on aina epänegatiivinen.
	\item Juuri on määritelty kaikilla luvuilla $a$, koska kaikkien lukujen neliöt ovat vähintään nolla. Vastaus on aina ei-positiivinen, koska $\sqrt[4]{a^2}$ on aina epänegatiivinen.
\end{enumerate}
\end{vastaus}
\end{tehtava}

\begin{tehtava}
Onko annettu juuri määritelty kaikilla luvuilla $a$? Millaisia arvoja juuri voi saada luvusta $a$ riippuen?\\
a) $\sqrt[5]{a^2}$ \quad b) $\sqrt[5]{a^3}$ \quad c) $\sqrt[5]{-a^2}$ \quad d) $- \sqrt[5]{a^2}$

\begin{vastaus}

\begin{enumerate}[a)]
	\item Juuri on määritelty kaikilla luvuilla $a$. Vastaus on aina epänegatiivinen.
	\item Juuri on määritelty kaikilla luvuilla $a$. Vastaus voi olla mikä tahansa luku.
	\item Juuri on määritelty kaikilla luvuilla $a$. Vastaus on aina ei-positiivinen.
	\item Juuri on määritelty kaikilla luvuilla $a$. Vastaus on aina ei-positiivinen.
\end{enumerate}
\end{vastaus}
\end{tehtava}

\begin{tehtava}
Sievennä $(3+\sqrt{3}x)^4:(\sqrt{3}+x)^3$.
\begin{vastaus}
$3 + \sqrt{3}x$
\end{vastaus}
\end{tehtava}

\begin{tehtava}
(YO 1888 1) Mikä on a/b:n arvo, jos \\
$ (\sqrt{a}+\sqrt{b}):(\sqrt{a}-\sqrt{b})=\sqrt{2}$ ?
\begin{vastaus}
$(3-2\sqrt{2}):(3+2\sqrt{2})$
\end{vastaus}
\end{tehtava}
